\documentclass[12pt,a4paper]{article}

\usepackage[czech]{babel}
\usepackage{csquotes}
\usepackage{graphicx}
\usepackage{textcomp}
% rovnice zarovnávat doleva
\usepackage[fleqn]{amsmath}

% neodsazovat nové odstavce
\setlength{\parindent}{0pt}


\begin{document}

%%%%%%%%%%%%%%%% TITLE PAGE %%%%%%%%%%%%%%%%
	\begin{titlepage}
		\begin{center}
			\includegraphics[width=0.5\linewidth]{img/logo.pdf}
			\vspace{3cm}
			
			\LARGE{Počítačové komunikace a sítě 2021/2022}
			\vspace{1cm}
			
			\LARGE\textbf{HTTP server}
			
			\vspace*{\fill}
			\large{Ondřej Mach (xmacho12)}
		\end{center}
	\end{titlepage}
	
	
%%%%%%%%%%%%%%%% THE ACTUAL DOCUMENT %%%%%%%%%%%%%%%%

	\section{Základy}
		Pro načtení signálu byl použit modul \texttt{wavfile} z knihovny \texttt{scipy.io}.
		Načtený signál má \textbf{45056 vzorků}.
		Počet vzorků vydělíme vzorkovací frekvencí a získáme tím délku signálu \textbf{2.816 s}.
		Hodnoty signálu se pohybují v rozmezí od \textbf{-1909} do \textbf{3584}.
	
	
	\section{Předzpracování a rámce}
		Signál je neprve ustředněn a normalizován.
		
		\begin{verbatim}
			data = data - data.mean()
			scale = np.abs(data).max()
			data = data / scale
		\end{verbatim}
		
		Dále je signál rozdělen na kousky o velikosti 512 pomocí metody \texttt{np.ndarray.resize}. 
		Je vytvořena prázdná matice \texttt{frames}, ve které každý sloupec bude obsahovat jeden rámec.
		Nakonec for cyklus projde všechny rámce a do každého zapíše konkatenaci dvou odpovídajících kousků.
		
		
	
	
\end{document}


















































